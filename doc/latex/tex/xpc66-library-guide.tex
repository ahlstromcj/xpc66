%-------------------------------------------------------------------------------
% xpc66-reference-guide
%-------------------------------------------------------------------------------
%
% \file        xpc66-reference-guide.tex
% \library     Documents
% \author      Chris Ahlstrom
% \date        2024-04-15
% \update      2024-04-15
% \version     $Revision$
% \license     $XPC_GPL_LICENSE$
%
%     This document provides LaTeX documentation for the xpc66 library.
%
%-------------------------------------------------------------------------------

\documentclass[
 11pt,
 twoside,
 a4paper,
 final                                 % versus draft
]{article}

\input{tex/docs-structure}             % specifies document structure and layout

\usepackage{fancyhdr}
\pagestyle{fancy}
\fancyhead{}
\fancyfoot{}
\fancyheadoffset{0.005\textwidth}
\lhead{Xpc66 Library Guide}
\chead{}
\rhead{Developer Guide}
\lfoot{}
\cfoot{\thepage}
\rfoot{}

% Removes the many "headheight is too small" warnings.

\setlength{\headheight}{14.0pt}

\makeindex

\begin{document}

\title{Xpc66 Developer Guide 0.1.0}
\author{Chris Ahlstrom \\
   (\texttt{ahlstromcj@gmail.com})}
\date{\today}
\maketitle

\begin{figure}[H]
   \centering 
   \includegraphics[scale=0.40]{xpc66.png}
   \caption*{Xpc66 Logo}
\end{figure}

\clearpage                             % moves Contents to next page

\tableofcontents
\listoffigures                         % print the list of figures

% No tables as of now
%
% \listoftables                        % print the list of tables

\parindent 0pt
\parskip 9pt

\rhead{\rightmark}         % shows section number and section name

\section{Introduction}
\label{sec:introduction}

   The \textsl{Xpc66} library reworks some of the fundamental code
   from the \textsl{Seq66} project (\cite{seq66}.
   This work is in preparation for the version 2 of that project.

   Cfg66 contains the following subdirectories of \texttt{src} and
   \texttt{include}, each of which holds modules in a
   namespace of the same name:

   \begin{itemize}
      \item \texttt{xpc}.
         Contains functions for daemonization, message handling,
         string manipulation, and file manipulation.
   \end{itemize}

   In the sections that follow, the basic are described.
   At some point we will make the effort to add some \textsl{Dia}
   diagrams to make the relationships more clear.

\subsection{Naming Conventions}
\label{subsec:introduction_conventions}

   \textsl{Xpc66} uses some conventions for naming things in this
   document.

   \begin{itemize}
      \item \texttt{\$prefix}. The base location for installation of
         the application and its ancillary data files on
         \textsl{UNIX/Linux/BSD}:
         \begin{itemize}
            \item \texttt{/usr/}
            \item \texttt{/usr/local/}
         \end{itemize}
      \item \texttt{\$winprefix}. The base location for installation of
         the application and its ancillary data files on \textsl{Windows}.
         \begin{itemize}
            \item \texttt{C:/Program Files/}
            \item \texttt{C:/Program Files (x86)/}
         \end{itemize}
      \item \texttt{\$home}. The location of the user's configuration files.
         Not to be confused with \texttt{\$HOME}, this is
         the standard location for configuration files.
         On a UNIX-style system, it would be \linebreak
         \texttt{\$HOME/.config/appname}.
         The files would be put into a \texttt{po} subdirectory here.
      \item \texttt{\$winhome}. This location is different for
         \textsl{Windows}:
         \texttt{C:/Users/user/AppData/Local/PACKAGE}.
   \end{itemize}

\subsection{Future Work}
\label{subsec:introduction_future}

   \begin{itemize}
      \item Hammer on this code in \textsl{Windows}.
   \end{itemize}

%----------------------------------------------------------------------------
% Additional Chapters
%----------------------------------------------------------------------------

%-------------------------------------------------------------------------------
% xpc
%-------------------------------------------------------------------------------
%
% \file        xpc.tex
% \library     Documents
% \author      Chris Ahlstrom
% \date        2024-04-16
% \update      2024-04-16
% \version     $Revision$
% \license     $XPC_GPL_LICENSE$
%
%     Provides a description of the entities in the xpc66 library.
%
%-------------------------------------------------------------------------------

\section{Xpc Namespace}
\label{sec:xpc_namespace}

   This section provides a useful walkthrough of the \texttt{xpc} namespace of
   the \textsl{xpc66} library.
   In addition, a \texttt{C}-only module is provided.

   Here are the classes (or modules) in this namespace:

   \begin{itemize}
      \item \texttt{automutex}
      \item \texttt{condition}
      \item \texttt{daemonize}
      \item \texttt{recmutex}
      \item \texttt{ring\_buffer}
      \item \texttt{shellexecute}
      \item \texttt{timing}
      \item \texttt{utilfunctions}
   \end{itemize}

\subsection{xpc::automutex}
\label{subsec:xpc_namespace_automutex}

   \texttt{xpc::automutex} provides a recursive
   mutex that locks automatically when
   created, and unlocks when destroyed.  This has a couple of benefits.
   First, it is threadsafe in the face of exception handling.
   Secondly, locking can be done with just one line of code.

   It could potentially be replaced by
   \texttt{std::lock\_guard<std::recursive\_mutex>}.
   One reason we rolled our own was some difficulty experienced using
   the standard mutex in the \textsl{Seq66} (\cite{seq66}) application.

   The constructor takes a reference to an \texttt{xpc::recmutex}
   (see below), stores it, and locks it.
   The destructor simply unlocks it.

\subsection{xpc::condition}
\label{subsec:xpc_namespace_condition}

   \texttt{xpc::condition} provides an internal recursive mutex and a
   private implementation of the \texttt{wait()} and
   \texttt{signal()} functions.
   The implementation uses a
   \texttt{pthread\_cond\_t} condition variable and
   \texttt{xpc::recmutex} to implement these functions.

   Also provided is the more useful and simpler
   \texttt{xpc::synchronizer} \textsl{abstract base class} which uses
   an \texttt{std::mutex} and
   an \texttt{std::condition\_variable}
   to implement the \texttt{wait()} and \texttt{signal()} functions.
   It requires the caller to derive a class which implements the
   \textsl{virtual} function \texttt{predicate()} that decides
   when synchronization has occurred.

   For a good example of \texttt{xpc::synchronizer}, see
   the \texttt{seq66::performer::synch} class defined
   in the \texttt{performer} module of the \textsl{Seq66} project.

\subsection{xpc::daemonize}
\label{subsec:xpc_namespace_daemonize}

   This module implements demonization code as described in
   \textsl{The Linux Programming Interface} (\cite{lpi}).
   It provides many options as expressed by the \texttt{daemonize\_flags}
   enumeration:

   \begin{itemize}
      \item Don't chdir() to the file root directory '/'.
      \item Don't close all open files.
      \item No stdin etc. sent to /dev/null.
      \item Don't call umask(0).
      \item Don't call fork() a second time.
      \item Don't change current directory.
      \item Do not open a system log file.
   \end{itemize}

   The most important functions are
   \texttt{daemonize()} and
   \texttt{undaemonize()}.
   For the usage of these functions, see the main module
   \texttt{seq66rtcli} in the \textsl{Seq66} project.

   Also provided in this module are functions for getting process information,
   rerouting standard I/O, and flagging session saving, restart, and closing.

   Note that this is a \texttt{C++}-only module using
   \texttt{std::string} to pass and store information.

\subsection{xpc::recmutex}
\label{subsec:xpc_namespace_recmutex}

   This recursive mutex is implemented using
   \texttt{pthread\_mutex\_t} due to difficulties we had
   with \texttt{C++11}'s \texttt{std::mutex} in the \textsl{Seq66} project.

   Read the module's comments for more information on the ifs, ands, buts, or
   maybes..

\subsection{xpc::ring\_buffer}
\label{subsec:xpc_namespace_ring_buffer}

   This template class defines a flexible ring-buffer.
   It support reading and writing, skipping, and the
   \texttt{front()} and
   \texttt{back()} functions.

   The \texttt{ring\_buffer.cpp} file contains an explanation of the
   implementation and some code to test the ring-buffer.

\subsection{xpc::shellexecute}
\label{subsec:xpc_namespace_shellexecute}

   This module provides free functions in the \texttt{xpc} namespace
   for spawning applications and opening PDFs and URLs.
   These functions provide support for
   \textsl{Linux/UNIX} and \texttt{Windows}.

   \begin{verbatim}
      command_line (const std::string & cmdline)
      open_document (const std::string & name)
      open_pdf (const std::string & pdfspec)
      open_url (const std::string & pdfspec)
      open_local_url (const std::string & pdfspec)
   \end{verbatim}

\subsection{xpc::timing}
\label{subsec:xpc_namespace_timing}

   This module provides free functions in the \texttt{xpc} namespace
   for getting the system time and for sleeping.
   These functions provide support for
   \textsl{Linux/UNIX} and \texttt{Windows}.

   \begin{verbatim}
      std_sleep_us ()
      microsleep (int us)
      millisleep (int ms)
      thread_yield ()
      microtime ()
      millitime ()
      set_thread_priority (std::thread & t, int p)
      set_timer_services (bool on)
   \end{verbatim}

   More explanation can be found in \texttt{timing.cpp}.

\subsection{xpc::utilfunctions}
\label{subsec:xpc_namespace_utilfunctions}

   In order to keep the \textsl{Xpc66 library} independent
   of the \textsl{Cfg66 library}, this module
   provides cut-down versions of the message functions of
   the latter.
   It also uses some code to work with directories,
   getting the date/time, and widening ASCII strings.

%-------------------------------------------------------------------------------
% vim: ts=3 sw=3 et ft=tex
%-------------------------------------------------------------------------------

%-------------------------------------------------------------------------------
% tests
%-------------------------------------------------------------------------------
%
% \file        tests.tex
% \library     Documents
% \author      Chris Ahlstrom
% \date        2024-04-17
% \update      2024-04-17
% \version     $Revision$
% \license     $XPC_GPL_LICENSE$
%
%     Provides a pointed description of some tests, which also helps understand
%     usage of the xpc66 library.
%
%-------------------------------------------------------------------------------

\section{Xfg66 Tests}
\label{sec:xpc66_tests}

   This section provides a useful walkthrough of the testing
   of the \textsl{xpc66} library.
   They illustrate the various ways in which the \textsl{Xfg66} library
   can be used by a developer.

   The tests so far are these executables:

    \begin{itemize}
        \item \texttt{xpc\_tests}
    \end{itemize}

    These tests are supported by data structures define in the following
    header files:

    \begin{itemize}
        \item texttt{todo.hpp}
    \end{itemize}

    These header files will be discussed as needed in the following sections.

\subsection{Xfg66 Test}
\label{subsec:xpc66_tests_test}

   TO DO.

   Obviously, we still have a lot of work to do with these tests.

%-------------------------------------------------------------------------------
% vim: ts=3 sw=3 et ft=tex
%-------------------------------------------------------------------------------


\section{Summary}
\label{sec:summary}

   Contact: If you have ideas about \textsl{Xpc66} or a bug report,
   please email us (at \url{mailto:ahlstromcj@gmail.com}).

% References

%-------------------------------------------------------------------------------
% references
%-------------------------------------------------------------------------------
%
% \file        references.tex
% \library     Documents
% \author      Chris Ahlstrom
% \date        2024-04-15
% \update      2024-04-15
% \version     $Revision$
% \license     $XPC_GPL_LICENSE$
%
%     Provides the References section of the Cfg66 manual. Rather
%     than use the bibtex package, our small set of references uses a
%     simpler method.
%     WRK
%
%-------------------------------------------------------------------------------

\section{References}
\label{sec:references}

   The \textsl{Cfg66} reference list.

{\RaggedRight
\begin{thebibliography}{99}

   \bibitem{lpi}
   Michael Kerrisk.
   \emph{The Linux Programming Interface}
   \url{https://man7.org/tlpi/}
   2010.

   \bibitem{seq66}
   Chris Ahlstrom.
   \emph{A reboot of the Seq24 project as "Seq66".}
   \url{https://github.com/ahlstromcj/seq66/}.
   2015-2024.

\end{thebibliography}
}

%-------------------------------------------------------------------------------
% vim: ts=3 sw=3 et ft=tex
%-------------------------------------------------------------------------------


\printindex

\end{document}

%-------------------------------------------------------------------------------
% vim: ts=3 sw=3 et ft=tex
%-------------------------------------------------------------------------------
