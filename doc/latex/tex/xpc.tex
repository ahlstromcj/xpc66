%-------------------------------------------------------------------------------
% xpc
%-------------------------------------------------------------------------------
%
% \file        xpc.tex
% \library     Documents
% \author      Chris Ahlstrom
% \date        2024-04-16
% \update      2024-04-16
% \version     $Revision$
% \license     $XPC_GPL_LICENSE$
%
%     Provides a description of the entities in the xpc66 library.
%
%-------------------------------------------------------------------------------

\section{Xpc Namespace}
\label{sec:xpc_namespace}

   This section provides a useful walkthrough of the \texttt{xpc} namespace of
   the \textsl{xpc66} library.
   In addition, a \texttt{C}-only module is provided.

   Here are the classes (or modules) in this namespace:

   \begin{itemize}
      \item \texttt{automutex}
      \item \texttt{condition}
      \item \texttt{daemonize}
      \item \texttt{recmutex}
      \item \texttt{ring\_buffer}
      \item \texttt{shellexecute}
      \item \texttt{timing}
      \item \texttt{utilfunctions}
   \end{itemize}

\subsection{xpc::automutex}
\label{subsec:xpc_namespace_automutex}

   \texttt{xpc::automutex} provides a recursive
   mutex that locks automatically when
   created, and unlocks when destroyed.  This has a couple of benefits.
   First, it is threadsafe in the face of exception handling.
   Secondly, locking can be done with just one line of code.

   It could potentially be replaced by
   \texttt{std::lock\_guard<std::recursive\_mutex>}.
   One reason we rolled our own was some difficulty experienced using
   the standard mutex in the \textsl{Seq66} (\cite{seq66}) application.

   The constructor takes a reference to an \texttt{xpc::recmutex}
   (see below), stores it, and locks it.
   The destructor simply unlocks it.

\subsection{xpc::condition}
\label{subsec:xpc_namespace_condition}

   \texttt{xpc::condition} provides an internal recursive mutex and a
   private implementation of the \texttt{wait()} and
   \texttt{signal()} functions.
   The implementation uses a
   \texttt{pthread\_cond\_t} condition variable and
   \texttt{xpc::recmutex} to implement these functions.

   Also provided is the more useful and simpler
   \texttt{xpc::synchronizer} \textsl{abstract base class} which uses
   an \texttt{std::mutex} and
   an \texttt{std::condition\_variable}
   to implement the \texttt{wait()} and \texttt{signal()} functions.
   It requires the caller to derive a class which implements the
   \textsl{virtual} function \texttt{predicate()} that decides
   when synchronization has occurred.

   For a good example of \texttt{xpc::synchronizer}, see
   the \texttt{seq66::performer::synch} class defined
   in the \texttt{performer} module of the \textsl{Seq66} project.

\subsection{xpc::daemonize}
\label{subsec:xpc_namespace_daemonize}

   This module implements demonization code as described in
   \textsl{The Linux Programming Interface} (\cite{lpi}).
   It provides many options as expressed by the \texttt{daemonize\_flags}
   enumeration:

   \begin{itemize}
      \item Don't chdir() to the file root directory '/'.
      \item Don't close all open files.
      \item No stdin etc. sent to /dev/null.
      \item Don't call umask(0).
      \item Don't call fork() a second time.
      \item Don't change current directory.
      \item Do not open a system log file.
   \end{itemize}

   The most important functions are
   \texttt{daemonize()} and
   \texttt{undaemonize()}.
   For the usage of these functions, see the main module
   \texttt{seq66rtcli} in the \textsl{Seq66} project.

   Also provided in this module are functions for getting process information,
   rerouting standard I/O, and flagging session saving, restart, and closing.

   Note that this is a \texttt{C++}-only module using
   \texttt{std::string} to pass and store information.

\subsection{xpc::recmutex}
\label{subsec:xpc_namespace_recmutex}

   This recursive mutex is implemented using
   \texttt{pthread\_mutex\_t} due to difficulties we had
   with \texttt{C++11}'s \texttt{std::mutex} in the \textsl{Seq66} project.

   Read the module's comments for more information on the ifs, ands, buts, or
   maybes..

\subsection{xpc::ring\_buffer}
\label{subsec:xpc_namespace_ring_buffer}

   This template class defines a flexible ring-buffer.
   It support reading and writing, skipping, and the
   \texttt{front()} and
   \texttt{back()} functions.

   The \texttt{ring\_buffer.cpp} file contains an explanation of the
   implementation and some code to test the ring-buffer.

\subsection{xpc::shellexecute}
\label{subsec:xpc_namespace_shellexecute}

   This module provides free functions in the \texttt{xpc} namespace
   for spawning applications and opening PDFs and URLs.
   These functions provide support for
   \textsl{Linux/UNIX} and \texttt{Windows}.

   \begin{verbatim}
      command_line (const std::string & cmdline)
      open_document (const std::string & name)
      open_pdf (const std::string & pdfspec)
      open_url (const std::string & pdfspec)
      open_local_url (const std::string & pdfspec)
   \end{verbatim}

\subsection{xpc::timing}
\label{subsec:xpc_namespace_timing}

   This module provides free functions in the \texttt{xpc} namespace
   for getting the system time and for sleeping.
   These functions provide support for
   \textsl{Linux/UNIX} and \texttt{Windows}.

   \begin{verbatim}
      std_sleep_us ()
      microsleep (int us)
      millisleep (int ms)
      thread_yield ()
      microtime ()
      millitime ()
      set_thread_priority (std::thread & t, int p)
      set_timer_services (bool on)
   \end{verbatim}

   More explanation can be found in \texttt{timing.cpp}.

\subsection{xpc::utilfunctions}
\label{subsec:xpc_namespace_utilfunctions}

   In order to keep the \textsl{Xpc66 library} independent
   of the \textsl{Cfg66 library}, this module
   provides cut-down versions of the message functions of
   the latter.
   It also uses some code to work with directories,
   getting the date/time, and widening ASCII strings.

%-------------------------------------------------------------------------------
% vim: ts=3 sw=3 et ft=tex
%-------------------------------------------------------------------------------
